\documentclass{article}

\usepackage[a4paper, left=1.5cm, right=1.5cm, top=1.5cm, bottom=1.5cm]{geometry}
\usepackage{amsmath}
\usepackage{color}
\usepackage{tikz}
\usepackage[utf8]{inputenc}
\usepackage[T1]{fontenc}
\usepackage{wasysym}
\usepackage[english]{babel}
\usepackage{titlesec}
\pagestyle{empty}

\definecolor{color_black}{rgb}{0,0,0}
\definecolor{color_red}{rgb}{1,0,0}

\begin{document}








\begin{flushright}




    % Replace 1 with the actual experiment number
    {\fontsize{12}{12}\selectfont \textmd 
    Experiment No: 1
    } \\



    \vspace{0.16cm}



    % just left some space for writing the date manually
    {\fontsize{12}{12}\selectfont \textmd 
    Date:  \hspace{2.08cm}
    }




\end{flushright}








\vspace{1.5cm}








\begin{center}




    % Enter the name of the experiment below
    {\fontsize{24}{28}\selectfont \bfseries 
     POLYNOMIAL ADDITION
    } \\



    
    \vspace{1.5cm}



    
\end{center}








\vspace{1.8cm}








% Aim Section
% Enter the aim of the experiment below
\noindent {\large \textbf{Aim:}} \\ \\
To read two polynomials and store them in an array. Calculate the sum of the two polynomials.








\vspace{1cm}








% Algorithm Section
% Enter the algorithm of the experiment below
\noindent {\large \textbf{Algorithm:}}  \\
\begin{enumerate}

    \item Start

    \item Some kind of description if needed
    \begin{verbatim}
      % Enter the Pseudo Code here
    \end{verbatim}

    \item Stop
    
\end{enumerate}
% The numbering of steps will be added automatically for each \item








\vspace{1cm}








% Hint!
% Add \newpage if you need the result in a new page.








% Result Section
% Enter the result of the experiment below
\noindent {\large \textbf{Result:}} \\ \\
Program has been executed successfully and obtained the output








\end{document}
