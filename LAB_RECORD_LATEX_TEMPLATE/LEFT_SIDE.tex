\documentclass{article}

\usepackage[a4paper, margin=1.5cm]{geometry}
\usepackage[T1]{fontenc}
\usepackage[utf8]{inputenc}
\usepackage{lmodern}
\usepackage{minted}
\usepackage{titlesec}
\usepackage{nopageno}

\titleformat{\section}[block]{\normalfont\Large\bfseries}{\thesection}{1em}{}[\titlerule]

\begin{document}








% <-- Program Section --> 
% Remove the comment and past the program code inside. Try using formated code for better reading. 
% You can do it here https://codebeautify.org/c-formatter-beautifier
\section*{Program}
\begin{minted}{c}

#include <stdio.h>

struct poly {
    float coeff;
    int exp;
};

struct poly p1[10], p2[10], p3[20];

int readpoly(struct poly p[]) {
    int t, i;
    printf("Enter number of terms in the polynomial: ");
    scanf("%d", &t);
    for (i = 0; i < t; i++) {
        printf("Enter coefficient of term %d: ", i + 1);
        scanf("%f", &p[i].coeff); 
        printf("Enter the exponent of the term %d: ", i + 1);
        scanf("%d", &p[i].exp);
    }
    printf("\n");
    return t;
}

int addpoly(struct poly p1[], struct poly p2[], struct poly p3[], int t1, int t2) {
    int i = 0, j = 0, k = 0;
    
    while (i < t1 && j < t2) {
        if (p1[i].exp == p2[j].exp) {
            p3[k].coeff = p1[i].coeff + p2[j].coeff;
            p3[k].exp = p1[i].exp;
            i++;
            j++;
        } else if (p1[i].exp > p2[j].exp) {
            p3[k] = p1[i];
            i++;
        } else {
            p3[k] = p2[j];
            j++;
        }
        k++;
    }

    while (i < t1) {
        p3[k++] = p1[i++];
    }
    
    while (j < t2) {
        p3[k++] = p2[j++];
    }
    
    return k; 
}

void displaypoly(struct poly p[], int t) {
    for (int k = 0; k < t; k++) {
        printf("%0.2f(x^%d)", p[k].coeff, p[k].exp);
        if (k < t - 1) { 
            printf(" + ");
        }
    }
    printf("\n");
}

int main() {
    int t1, t2, t3;

    printf("Polynomial 1\n");
    t1 = readpoly(p1);
    printf("Polynomial 1: ");
    displaypoly(p1, t1); 

    printf("Polynomial 2\n");
    t2 = readpoly(p2);
    printf("Polynomial 2: ");
    displaypoly(p2, t2); 

    printf("\nAdded polynomial:\n");
    t3 = addpoly(p1, p2, p3, t1, t2);
    
    displaypoly(p3, t3); 
    
    return 0;
}

\end{minted}








% <-- Output Section --> 
% Remove the comment and past the output inside.
\section*{Output}
\begin{verbatim}

gcc PolynomialAddition.c
./a.out
Polynomial 1
Enter number of terms in the polynomial: 4
Enter coefficient of term 1: 6
Enter the exponent of the term 1: 8
Enter coefficient of term 2: 3
Enter the exponent of the term 2: 3
Enter coefficient of term 3: 2
Enter the exponent of the term 3: 2
Enter coefficient of term 4: 5
Enter the exponent of the term 4: 0

Polynomial 1: 6.00(x^8) + 3.00(x^3) + 2.00(x^2) + 5.00(x^0)
Polynomial 2
Enter number of terms in the polynomial: 3
Enter coefficient of term 1: 5
Enter the exponent of the term 1: 10
Enter coefficient of term 2: 3
Enter the exponent of the term 2: 8
Enter coefficient of term 3: 4
Enter the exponent of the term 3: 3

Polynomial 2: 5.00(x^10) + 3.00(x^8) + 4.00(x^3)

Added polynomial:
5.00(x^10) + 9.00(x^8) + 7.00(x^3) + 2.00(x^2) + 5.00(x^0)

\end{verbatim}









\end{document}