\documentclass{article}

\usepackage[a4paper, left=1.5cm, right=1.5cm, top=1.5cm, bottom=1.5cm]{geometry} % Adjust margins
\usepackage{amsmath}
\usepackage{color}
\usepackage{tikz}
\usepackage[utf8]{inputenc}
\usepackage[T1]{fontenc}
\usepackage{wasysym}
\usepackage[english]{babel}
\usepackage{titlesec}
\pagestyle{empty}

\definecolor{color_black}{rgb}{0,0,0}
\definecolor{color_red}{rgb}{1,0,0}

\begin{document}





\begin{flushright}



    % Type the Experiment Number here below 
    {\fontsize{12}{12}\selectfont \textmd 
    Experiment No: 1
    } \\



    \vspace{0.16cm}



    % Type the Date here below 
    {\fontsize{12}{12}\selectfont \textmd 
    Date:  \hspace{2.08cm}
    }
\end{flushright}





\vspace{1.5cm}





\begin{center}



    % Type the Experiment name here
    {\fontsize{24}{28}\selectfont \bfseries 
     POLYNOMIAL ADDITION
    } \\


    
    \vspace{1.5cm}


    
\end{center}





\vspace{1.8cm}





% Aim Section
% Type the Aim below 
\noindent {\large \textbf{Aim:}} \\ \\
To read two polynomials and store them in an array. Calculate the sum of the two polynomials.





\vspace{1cm}





% Algorithm Section
% Type the Algorithm below 
\noindent {\large \textbf{Algorithm:}}  \\
\begin{enumerate}
    \item Start

    \item Create a structure poly (float coeff, int exp)

    \item Create array of structure p1[10], p2[10], p3[10]

    \item Define function int readpoly (structure poly p[])
    
    \item Read number of terms in the polynomial - t1

    \item Begin for loop from i=0 to t1 \\ read coefficients and exponents \\ return t1
    
    \item Define function addpoly (structure poly p1[10], structure poly p2[10], structure poly p3[10], int t1, int t2)
    
    \item Initialize i, j, k to 0

    \item Begin while loop: 
    \begin{verbatim}
      i < t1 & j < t2
      if    ( p1[i].exp == p2[j].exp )
            p3[k].coeff = p1[i].coeff + p2[j].coeff
            p3[k].exp = p1[i].exp
            i++, j++, k++
      else if    ( p1[i].exp > p2[j].exp )
            p3[k].coeff = p1[i].coeff
            p3[k].exp = p1[i].exp
            i++, k++
      else  p3[k].coeff = p2[j].coeff
            p3[k].exp = p2[j].exp
            j++, k++
    \end{verbatim}

    \item Begin while loop : i < t1
    \begin{verbatim}
      p3[k].coeff = p1[i].coeff
      p3[k].exp = p1[i].exp
      i++, k++
    \end{verbatim}

    \item Begin while loop : j < t2
    \begin{verbatim}
      p3[k].coeff = p2[i].coeff
      p3[k].exp = p2[i].exp
      j++, k++
    \end{verbatim}

    \item Define function displaypoly (structure poly p[], int t) \\
    Begin for loop from k = 0 to t-1 \\
    display polynomial

      
    \item In main ()
     \begin{verbatim}
      Call readpoly(p1)
      Call displaypoly(p1, t1)
      Call readpoly(p2)
      Call displaypoly(p2, t2)
      Call addpoly(p1, p2, p3, t1, t2)
      Call displaypoly(p3, t3)
     \end{verbatim}

    \item Stop
    
\end{enumerate}





\vspace{1cm}





% \newpage
% Add \newpage if you need the result in a new page.





% Result Section
\noindent {\large \textbf{Result:}} \\ \\
Program has been executed successfully and obtained the output





\end{document}